%%%%%%%%%%%%%%%%%%%%%%%%%%%%%%%%%%%%%%%%%
% Jacobs Landscape Poster
% LaTeX Template
% Version 1.1 (14/06/14)
%
% Created by:
% Computational Physics and Biophysics Group, Jacobs University
% https://teamwork.jacobs-university.de:8443/confluence/display/CoPandBiG/LaTeX+Poster
% 
% Further modified by:
% Nathaniel Johnston (nathaniel@njohnston.ca)
%
% This template has been downloaded from:
% http://www.LaTeXTemplates.com
%
% License:
% CC BY-NC-SA 3.0 (http://creativecommons.org/licenses/by-nc-sa/3.0/)
%
%%%%%%%%%%%%%%%%%%%%%%%%%%%%%%%%%%%%%%%%%

%----------------------------------------------------------------------------------------
%	PACKAGES AND OTHER DOCUMENT CONFIGURATIONS
%----------------------------------------------------------------------------------------

\documentclass[final]{beamer}

\usepackage[scale=1.0]{beamerposter} % Use the beamerposter package for laying out the poster
\usetheme{confposter} % Use the confposter theme supplied with this template

\setbeamercolor{block title}{fg=dblue!80,bg=white} % Colors of the block titles
\setbeamercolor{block body}{fg=black,bg=white} % Colors of the body of blocks
\setbeamercolor{block alerted title}{fg=white,bg=dblue!70} % Colors of the highlighted block titles
\setbeamercolor{block alerted body}{fg=black,bg=dblue!10} % Colors of the body of highlighted blocks
% Many more colors are available for use in beamerthemeconfposter.sty

%-----------------------------------------------------------
% Define the column widths and overall poster size
% To set effective sepwid, onecolwid and twocolwid values, first choose how many columns you want and how much separation you want between columns
% In this template, the separation width chosen is 0.024 of the paper width and a 4-column layout
% onecolwid should therefore be (1-(# of columns+1)*sepwid)/# of columns e.g. (1-(4+1)*0.024)/4 = 0.22
% onecolwid should therefore be (1-(# of columns+1)*sepwid)/# of columns e.g. 
% (1-(3+1)*0.025)/3 = 0.3
% Set twocolwid to be (2*onecolwid)+sepwid = 0.464
% Set threecolwid to be (3*onecolwid)+2*sepwid = 0.708

\newlength{\sepwid}
\newlength{\onecolwid}
\newlength{\twocolwid}
\newlength{\threecolwid}
\setlength{\paperwidth}{36in} % A0 width: 46.8in
\setlength{\paperheight}{48in} % A0 height: 33.1in
\setlength{\textwidth}{34in} % A0 width: 46.8in
\setlength{\textheight}{46in} % A0 height: 33.1in
\setlength{\sepwid}{0.025\paperwidth} % Separation width (white space) between columns
\setlength{\onecolwid}{0.3\paperwidth} % Width of one column
\setlength{\twocolwid}{0.625\paperwidth} % Width of two columns
\setlength{\threecolwid}{0.95\paperwidth} % Width of three columns
\setlength{\topmargin}{-0.5in} % Reduce the top margin size
%-----------------------------------------------------------

\usepackage{graphicx}  % Required for including images
\newcommand{\Cyclus}{\textsc{Cyclus}\xspace}%

\usepackage{tabularx}
\newcolumntype{b}{X}
\newcolumntype{s}{>{\hsize=.5\hsize}X}
\newcolumntype{m}{>{\hsize=.75\hsize}X}
\newcolumntype{z}{>{\hsize=.65\hsize}X}

\usepackage{booktabs} % Top and bottom rules for tables
\usepackage{xspace}

\usepackage{tikz}
\usetikzlibrary{positioning, arrows, decorations, shapes }
% Define block styles
\tikzstyle{decision} = [diamond, draw, fill=blue!20, 
text width=4.5em, text badly centered, node distance=3cm, inner sep=0pt]


\tikzstyle{block} = [rectangle, draw, text centered, fill=blue!20]
\tikzstyle{line} = [draw, -latex']
\tikzstyle{cloud} = [draw, ellipse,fill=red!20, node distance=6em,
minimum height=2em]



\usetikzlibrary{shapes.multipart}
\usetikzlibrary{positioning}


\setbeamertemplate{bibliography item}[text]

%----------------------------------------------------------------------------------------
%	TITLE SECTION 
%----------------------------------------------------------------------------------------

\title{
	\includegraphics[width=0.3\linewidth]{NC_State_Logo.png}
	\hspace{30cm}
	\vspace{2cm}
	\includegraphics[width=0.3\linewidth]{cnec_logo.png} \\
	Using Cyclus for Online Diversion Detection of Shadow Fuel Cycles
} % Poster title

\author{Gregory T. Westphal, Kathryn D. Huff}
\institute{University of Illinios at Urbana-Champaign, Department of Nuclear, Plasma, and Radiological Engineering, Urbana, IL 61801}
%----------------------------------------------------------------------------------------

\begin{document}

\addtobeamertemplate{block end}{}{\vspace*{2ex}} % White space under blocks
\addtobeamertemplate{block alerted end}{}{\vspace*{2ex}} % White space under highlighted (alert) blocks

\setlength{\belowcaptionskip}{2ex} % White space under figures
\setlength\belowdisplayshortskip{2ex} % White space under equations

\begin{frame}[t] % The whole poster is enclosed in one beamer frame

\begin{columns}[t,totalwidth=\threecolwid] % The whole poster consists of three major columns, the second of which is split into two columns twice - the [t] option aligns each column's content to the top

\begin{column}{\sepwid}\end{column} % Empty spacer column

\begin{column}{\onecolwid} % The first column

%----------------------------------------------------------------------------------------
%	OBJECTIVES
%----------------------------------------------------------------------------------------

\begin{alertblock}{Objectives}
Apply Cyclus for online diversion detection by simulating the fuel cycle of a pyroprocessing plant
and compare to false positive rate to previous methods.
\begin{itemize}
	\item Investigate promising signatures and observables.
	\item Review existing nuclear diversion detection.
	\item Identify successful algorithms in other fuel cycles.
\end{itemize}

\end{alertblock}

%----------------------------------------------------------------------------------------
%	BACKGROUND
%----------------------------------------------------------------------------------------

\begin{block}{Background}

Shown in Figure 1 is the expected fuel cycle, however, with each process there is opportunity for proliferation.
In transportation of conversion and enrichment, for example, excess can be diverted without attention. Other
scenarios such as unreported reactor operation, over-usage, or misuse \cite{boyer_2014}. However, for this work diversion will be
of primary focus.

\begin{figure}
	\includegraphics[width=1.0\linewidth]{fuel_cycle2.png}
	\caption{Typical Nuclear fuel cycle without diversion \cite{Slovenske}.}
\end{figure}

\end{block}

%----------------------------------------------------------------------------------------
%	SIGNATURES AND OBSERVABLES
%----------------------------------------------------------------------------------------

\begin{block}{Signatures and Observables}
        Detection modes vary between each facility type, requiring a specific analysis of each processing plant to determine
        signatures and observables resulting in highest success rate. Pyroprocessing facilities have four major systems with waste
        that can be tracked: electroreduction, electrorefining, electrowinning, and metal fuel fabrication\cite{Borrelli_2017}.
        These systems have the corresponding signatures:
	
	\begin{itemize}
		\item \textbf{Metal Waste:} Solid, insoluble metal fission products.
		\item \textbf{Ceramic Waste Electrowinning:} Waste salt LiCl-KCl contains trace amounts of $^{135}$Cs and $^{137}$Cs from
		electrowinning the fuel.
		\item \textbf{Vitrified Waste:} LiCl-KCl salt that contains TRU and Sr alongside rare-earth elements precipitated into gases
		and vitrified with borosilicate glass.
		\item \textbf{Ceramic Waste Electroreduction:} Through electroreduction, Li$_2$CO$_3$ is used to separate $^{135}$Cs, $^{137}$Cs, 
		$^{129}$I and $^{14}$C which are solidified into ceramic waste.
	\end{itemize}

		External analysis is also needed for detection of potential rogue nuclear fuel cycles, in which direct observation can not be
		utilized in most scenarios. For cases such as inspecting international fuel cycles or performing an audit on domestic networks
		without physical intervention, the following signatures are viable:
		
	\begin{itemize}
		\item \textbf{Power Draw:} Sign of overusing centrifuges \cite{Yilmaz_2016,Hou_2016}.
		\item \textbf{Smoke Production:} Reactor producing high power than rated or reported for possible
		nefarious reasons \cite{Yilmaz_2016}.
		\item \textbf{Decay Heat:} Lower decay heat in casks signifies over-reporting of waste \cite{Kemp_2016}.
		\item \textbf{Trace Quantities:} $^{135}$Xe and $^{85}$Kr are commonly emitted through processing along with tritium
		from reactors. Need sensitive equipment but difficult to hide \cite{Borrelli_2017,Kemp_2016}.
	\end{itemize}
	
\end{block}

%----------------------------------------------------------------------------------------

\end{column} % End of the first column

\begin{column}{\sepwid}\end{column} % Empty spacer column


%----------------------------------------------------------------------------------------

\begin{column}{\onecolwid} % The second column

%----------------------------------------------------------------------------------------
%	DIVERSION DETECTION
%----------------------------------------------------------------------------------------

\begin{block}{Diversion Detection}\end{block}

	\begin{figure}
		\includegraphics[width=0.9\linewidth]{Yilmaz_Graphic}
		\caption{Example nuclear fuel cycle with diversion element \cite{Yilmaz_2016}}
	\end{figure}

        \begin{alertblock}{PLACEHOLDER FOR SHADOW GRAPHIC}
	\begin{itemize}
%		\setlength\itemsep{1em}
		\item {\large Probabilistic search into the objective function or constraint models}
		\item {\large Uncertainty in addition to mean}
		\item {\large Uses random samples from probability distributions}
		\item {\large E.g. Markov Switching-Model, Gaussian Process Regression}
	\end{itemize}
        \end{alertblock}
    
%----------------------------------------------------------------------------------------
%	PREVIOUS WORK
%----------------------------------------------------------------------------------------

\begin{block}{Previous Work}
Two new approaches to online diversion detection have recently been proposed \cite{Hou_2016,Yilmaz_2016}. Both
approaches rely on the previously mentioned power demand signature. To facilitate online detection shipment transportation
properties are used as follows \cite{Yilmaz_2016}:

\begin{itemize}
		\item Product enrichment
		\item Frequency of shipments
		\item Time to production
	\end{itemize}
These properties are used to determine the likelihood of a shadow shipment. The first proposed method uses maximum likelihood
estimation to determine unreported routes of transport \cite{Hou_2016}. 
\begin{figure}
	\includegraphics{Hou_Network.png}
	\caption{Maximum Likelihood Estimation nodal representation proposed by Hou et al\cite{Hou_2016}.}
\end{figure}
Shown in Figure 4 above, is a visual representation of the nodal network setup for maximum likelihood. The second approach
similarly requires a known network but instead uses cumulative sum assuming a Poisson distribution \cite{Yilmaz_2016}.
By combining the properties listed above, expected values can be derived using distributions for each enrichment and
shipment speed. \\
\vspace{10mm}
Prior work also exists in prevention of shadow fuel cycles. Analysis of varying plant archetypes and their risk of theft/nefarious
actions are provided \cite{Rossi_2016}. Rossi concludes that changing the material type such that technical difficulty is increased
equates to the greatest decrease in risk.
\end{block}

%----------------------------------------------------------------------------------------

\end{column} % End of column 2

\begin{column}{\sepwid}\end{column} % Empty spacer column

\begin{column}{\onecolwid} % The third column
	
%----------------------------------------------------------------------------------------
%	FUTURE WORK
%----------------------------------------------------------------------------------------

\begin{block}{Future Work}
	The goal of this poster is to outline the ground work done and review previous material on diversion, particularly related
	to pyroprocessing. What needs to be accomplished proceeding this work is as follows:
	\begin{itemize}
		\item Simulate pyroprocessing plant and network.
		\item Create Cyclus output and compare to prior algorithms.
		\item Assess capability of using Cyclus as online detection.
	\end{itemize} 
	\vspace{10mm}
	A pyroprocessing plant must be simulated for the appropriate input data into Cyclus, with a focus on grid
	information. Previously mentioned observables such as decay heat or smoke production are not well suited for
	Cyclus \cite{Huff_2016}, therefore data such as power draw or possibly transportation times and frequency will be explored 
	instead \cite{Hou_2016}.
\end{block}

%----------------------------------------------------------------------------------------
%	ACKNOWLEDGEMENTS
%----------------------------------------------------------------------------------------

\setbeamercolor{block title}{fg=norange,bg=white} % Change the block title color

\begin{block}{Acknowledgements}
	
	This research was performed using funding received
	from the Consortium for Nonproliferation Enabling
	Capabilities under award number 1-483313-973000-191100.
	
	\vspace{10mm}
	\begin{center}
		\begin{tabular}{ccc}
			\includegraphics[width=0.3\linewidth]{logo.png} & \includegraphics[width=0.3\linewidth]{cnec_oldlogo.png}
		\end{tabular}
	\end{center}
	
	
\end{block}

%----------------------------------------------------------------------------------------
%	CONTACT INFORMATION
%----------------------------------------------------------------------------------------

\setbeamercolor{block alerted title}{fg=black,bg=norange} % Change the alert block title colors
\setbeamercolor{block alerted body}{fg=black,bg=white} % Change the alert block body colors



\begin{alertblock}{Contact Information}
	\setbeamercolor{block title}{fg=norange,bg=white} % Change the block title color
	\begin{itemize}
		
		\item Web: \href{arfc.github.io}{arfc.github.io}
		\item Email: \href{mailto:gtw2@illinois.edu}{gtw2@illinois.edu}
		\item Phone: +1 (636) 284-9691
	\end{itemize}
	
\end{alertblock}

\begin{block}{References}

        {\footnotesize\bibliographystyle{abbrv} 
        \bibliography{poster}}
\end{block}


%----------------------------------------------------------------------------------------



\end{column} % End of the third column

\end{columns} % End of all the columns in the poster

\end{frame} % End of the enclosing frame

\end{document}
\begin{column}{\sepwid}\end{column} % Empty spacer column
